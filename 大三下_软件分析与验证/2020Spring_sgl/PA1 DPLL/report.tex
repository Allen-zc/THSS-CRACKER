%% compile with pdflatex or xelatex
\documentclass[11pt,a4paper]{article}
\usepackage[fontset=ubuntu]{ctex}
\usepackage{homework}
\usepackage{CJKutf8}

\begin{CJK}{UTF8}{gkai}
\title{PA1 Report}
\duedate{Mar 17, 2020}


\studentname{沈冠霖}
\studentid{2017013569}

\usepackage{listings}
\lstset{basicstyle=\footnotesize\ttfamily}

\begin{document}

\maketitle
\paragraph{实现思路}
带backjump的DPLL整体上是在DPLL基础上实现的。相比用backtrack的DPLL,带backjump的DPLL只是添加了冲突图的维护,以及将backtrack替换成backjump。\\
backjump算法实现的基本思路和课件一样:维护一个冲突图,在冲突的时候对冲突图求割,然后进行对应的回退和添加对应的子句。\\
为了简化求割的步骤,减少开销,我将每次选的reason侧定义为这个连通子图的所有决策节点,选取的K节点集合也就是这个连通子图的所有决策节点。这样可以保证保证所有决策节点在reason侧,至少一个冲突文字在另一侧,而且选取的节点都有边连接另一侧。同时,这种选取方法能够大大简化求割的步骤。\\
这样实现的话,我的冲突图其实并不需要建成一个图,我只需要记录每个点的文字和它的根本祖先(产生它的决策节点集合)即可。遇到冲突的时候,把冲突的两个文字的根本祖先集合求并,就是选取的节点集合了。\\
\paragraph{测试环境}
CPU:Inter Core i5-6300HQ,2.3GHZ\\
内存:12G\\
环境:ubuntu18.04, release模式
\paragraph{测试结果}:虽然对于很多没有多少回溯的数据,backjump是更慢的,因为它有维护冲突图的开销,在backjump步骤中进行集合求并的开销也不小。但是对于回溯频繁的数据,backjump更快。\\
我选取了测试集合的三组数据,分别是test9.dimacs(a graph of 6 vertices with a 3-clique and a 4-coloring),test11.dimacs(Pigeonhole principle formula for 6 pigeons and 7 holes),test12.dimacs(5 vertices with a 3-clique and a 2-coloring)。对于这三组数据,我的backjump远远快于backtrack。\\
\begin{table}[!htbp] 
	
	\caption{backtrack与backjump在三组数据的结果比较(时间单位为ms)}
	\begin{flushleft} 
		\begin{tabular}{|l|l|l|l|l|l|l|l|l|} 
			\hline 数据&算法 & 1 & 2 & 3 & 4 & 5 & 平均\\ 
			\hline test9&backtrack &865.871&855.808&857.898&864.609&853.403&859.52 \\
			\hline &backjump &15.5159&14.7249&13.9436&14.8857&13.7598&14.57 \\  
			\hline test11&backtrack &21.4267&18.2216&16.9051&19.796&17.3519&18.74 \\
			\hline &backjump &7.19779&6.2324&6.7309&6.7755&8.06141&7.00 \\  
			\hline test12&backtrack &636.489&630.947&630.145&629.309&628.896&631.76 \\
			\hline &backjump &81.2241&78.3284&78.7136&79.3428&78.1387&79.15 \\ 
			\hline 
		\end{tabular} 
	三组数据,backjump的时间分别是backtrack的$\frac{1}{60},\frac{3}{8},\frac{1}{8}$,可以充分说明,对于这三个问题,backjump比backtrack快很多。
	\end{flushleft} 
\end{table}

\end{CJK}
\end{document}
