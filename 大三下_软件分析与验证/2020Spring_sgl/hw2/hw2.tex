%% compile with pdflatex or xelatex
\documentclass[11pt,a4paper]{article}
\usepackage[fontset=ubuntu]{ctex}
\usepackage{homework}

\title{Homework 2}
\duedate{Mar 24, 2020}

\studentname{沈冠霖}
\studentid{2017013569}

\usepackage{tikz}

\usepackage{listings}
\definecolor{mygray}{rgb}{0.5,0.5,0.5}
\definecolor{backgray}{gray}{0.95}
\lstdefinestyle{dafny}{ % dafny syntax highlight
	belowcaptionskip=1\baselineskip,
	breaklines=true,
	language=C,
	showstringspaces=false,
	numbers=left,
	xleftmargin=2em,
	framexleftmargin=1.5em,
	numbersep=5pt,
	numberstyle=\tiny\color{mygray},
	basicstyle=\small\ttfamily,
	keywordstyle=\color{blue},
	commentstyle=\itshape\color{purple!40!black},
	morekeywords={
		abstract, array, as, assert, assume, bool, break, calc, case, char, class, codatatype, colemma, constructor, copredicate, datatype, decreases, default, else, ensures, exists, extends, false, forall, free, fresh, function, ghost, if, imap, import, in, include, inductive, int, invariant, iset, iterator, label, lemma, map, match, method, modifies, modify, module, multiset, nat, new, newtype, null, object, old, opened, predicate, print, protected, reads, real, refines, requires, return, returns, seq, set, static, string, then, this, trait, true, type, var, where, while, yield, yields
	},
	tabsize=2,
	backgroundcolor=\color{backgray}
}
\lstset{style=dafny}

%% logical symbols
% \land     /\
% \lor      \/
% \lnor     ¬
% \to       ->
% \lequiv   <->
% \exists   ∃
% \forall   ∀
% \models   |=
\newcommand{\lequiv}{\leftrightarrow}
\newcommand{\nat}{\mathbb{N}}

\begin{document}

\maketitle

\textit{Read the instructions below carefully before you start working on the assignment:}
\begin{itemize}
    \item In this assignment, you are asked to both typeset your answers in the attached \LaTeX~source file, and also complete the missing code in the Dafny file \texttt{PA.dfy}.
    Make sure that the Dafny complier can type check your code!
    When done, compile this file to a PDF.
    Compress the PDF and \texttt{PA.dfy} to an \texttt{.zip} archive and hand it to Tsinghua Web Learning \emph{before the due date}.
    \item Make sure you fill in your \emph{name} and \emph{Tsinghua ID},
    and replace all ``\texttt{TODO}''s with your solutions.
    \item Any kind of dishonesty is \emph{strictly prohibited} in the full semester.
    If you refer to any material that is not provided by us, you \emph{must cite} it.
\end{itemize}

%% problem begins

\problem{Short-Answered Questions}

\subproblem Underline all free variables (to be precise, their occurrences) in the following first-order formula:
\begin{equation*}
    \forall x. (f(x) \land (\exists y. g(x, y, z))) \land (\exists z. g(x, y, z))
\end{equation*}

\begin{solution}
    \forall x. (f(x) \land (\exists y. g(x, y, \underline{z}))) \land (\exists z. g(x, \underline{y}, z))
\end{solution}

\subproblem Which of the following problems or theories are decidable?

\begin{enumerate}[label=(\alph*)]
    \item Deciding validity for propositional logic.
    \item Deciding validity for first-order logic.
    \item $T_{E}$.
    \item $T_{\nat}$.
    \item The quantifier-free fragment of $T_A$.
\end{enumerate}

\begin{solution}
    (a),(d),(e)
\end{solution}

\subproblem Find an equivalence relation that is not a congruence relation.

\begin{solution}
    对于集合S=\{a,b,c\},在S上定义函数f:f(a)=c,f(b)=b,f(c)=c,还有关系R:\{(a,a),(b,b),(a,b),(b,a),(c,c)\}\\
    首先,R是等价关系,可以划分为两个等价类\{a,b\}和\{c\}\\
    其次,R不是共轭关系。一方面,aRb成立;另一方面,f(a)=c,f(b)=b,f(a)Rf(b)不成立;\\
\end{solution}

\subproblem Find two distinct equivalence relations such that one refines the other.

\begin{solution}
    对于集合S=\{a,b,c\},在S上定义关系R_1\{(a,a),(b,b),(c,c)\},关系R_2:\{(a,a),(b,b),(a,b),(b,a),(c,c)\}\\
    一方面,$R_1,R_2$ 都是等价关系。另一方面,$\hat{R_1} \in \hat{R_2}, R_1\prec R_2 $
\end{solution}

\subproblem In the congruence closure algorithm, subterms of a formula are represented by DAGs.
Which term does \cref{fig:subterm} represents?
Write it out in a formulaic way.

\tikzstyle{node} = [draw, circle, text centered, minimum width=2em]
\tikzstyle{next} = [draw, ->]
\begin{figure}[ht]
    \begin{minipage}[b]{.45\textwidth}
        \centering
        \begin{tikzpicture}[every node/.style=node, every path/.style=next]
            \node {$f$}
                child {
                    node (f2) {$f$}
                    child {
                        node {$f$}
                        child {
                            node {$x$}
                        }
                        child {
                            node (y) {$y$}
                        }
                    }
                }
                child {
                    node {$z$}
                }
            ;
            \path (f2) to [bend left] (y);
        \end{tikzpicture}
        \caption{A subterm.}
        \label{fig:subterm}
    \end{minipage}
    \begin{minipage}[b]{.45\textwidth}
        \centering
        \begin{tikzpicture}[every node/.style=node, every path/.style=next]
            \node {$f$}
                child {
                    node {$f$}
                    child {
                        node (f3) {$f$}
                        child {
                            node {$f$}
                            child {
                                node (x) {$x$}
                            }
                        }
                    }
                }
            ;
        \path [draw, dashed] (f3) to [bend left] (x);
    \end{tikzpicture}
    \caption{A DAG.}
    \label{fig:dag}
    \end{minipage}
\end{figure}

\begin{solution}
f(f(f(x,y),y),z)
\end{solution}

\subproblem \cref{fig:dag} presents a DAG in an execution of the congruence closure algorithm.
The dashed edge was inserted via a union operation.
Which congruences classes can you infer from this figure?

\begin{solution}
   $\{x,f^{2}(x),f^{4}(x)\}$和$\{f^{1}(x),f^{3}(x)\}$两个共轭类。\\
   首先,有$\{x,f^{2}(x)\},\{f^{1}(x)\},\{f^{3}(x)\},\{f^{4}(x)\}$四个等价类。\\
   其次,根据共轭条件,若有$xRf^{2}(x)$,则必有$f(x)Rf(f^{2}(x))$,也即$f^{1}(x)R{f^{3}(x)$。同理,有$f^{2}(x)}Rf^{4}(x)$\\
   将两条新增关系加入原先的等价关系R,再次扩增R使其满足等价性和共轭条件,则能得到$\{x,f^{2}(x),f^{4}(x)\}$和$\{f^{1}(x),f^{3}(x)\}$两个共轭类。
\end{solution}

\newpage
\problem{Peano Arithmetic}

In this problem, we will show that two ways of defining natural numbers are, to some extent, the same -- one is using the axioms of Peano Arithmetic, and another is using an inductive set whose elements are what we mean ``natural numbers''.

To receive full credit of this problem, you must both:
\begin{itemize}
    \item complete the proofs in \texttt{PA.dfy}, and
    \item fill in the missing manual proofs in this file.
\end{itemize}
There is an example that has been done for you.
You should read it carefully before you start.

\paragraph*{PA}

Recall that \emph{Peano Arithmetic} (PA) is a first-order theory with signature:
\[
\Sigma_\mathit{PA}: \{0, 1, +, \times, =\}
\]
where:
\begin{itemize}
    \item $0$ and $1$ are constants
    \item $+$ and $\times$ are binary functions
    \item $=$ is a binary predicate
\end{itemize}

It has the following axioms:
\begin{itemize}
    \item All of the equality axioms: reflexivity, symmetry, transitivity, and congruence
    \item \emph{Zero:} $\forall x .~\lnot(x + 1 = 0)$
    \item \emph{Additive identity:} $\forall x .~x + 0 = x$
    \item \emph{Times zero:} $\forall x .~x \times 0 = 0$
    \item \emph{Successor:} $\forall x, y .~(x + 1 = y + 1) \to x = y$
    \item \emph{Plus successor:} $\forall x, y .~x + (y + 1) = (x + y) + 1$
    \item \emph{Times successor:} $\forall x, y .~x \times (y + 1) = x \times y + x$
\end{itemize}
It also has an axiom schema for induction:
\[
(F[0] \land (\forall x . F[x] \to F[x + 1])) \to \forall x . F[x]
\]

The intended interpretation for this theory is the natural numbers with constant symbols 0 and 1, a predicate symbol $=$ taking equality over $\nat$, and function symbols $+,\times$ taking the corresponding expected functions over $\nat$.

\paragraph*{Natural Numbers as an Inductive Set}

On the other hand, we could define natural numbers as an \emph{inductive} set $S$, i.e. a \emph{minimum} set whose elements are generated by using only the following two rules:
\begin{itemize}
    \item $0 \in S$;
    \item if $n \in S$, then $\textsf{Succ}(n) \in S$.
\end{itemize}
In fact, our familiar $\nat = S$ as defined above.
Using the above notion, 1 is represented by $\textsf{Succ}(0)$; 2 is represented by $\textsf{Succ}(\textsf{Succ}(0))$; and so on.

The above definition can be easily expressed in Dafny, as an inductive type:
\begin{lstlisting}
datatype Nat = Zero | Succ(n: Nat)
\end{lstlisting}
We then define the constant 1, as well as the functions for addition and multiplication as follows:
\begin{lstlisting}
function one(): Nat
{
  Succ(Zero)
}

function add(x: Nat, y: Nat): Nat
{
  match(x) {
    case Zero => y
    case Succ(n) => Succ(add(n, y))
  }
}

function mult(x: Nat, y: Nat): Nat
{
  match(x) {
    case Zero => Zero
    case Succ(n) => add(mult(n, y), y)
  }
}
\end{lstlisting}

In this problem, you will use Dafny to prove that the inductively-defined \texttt{Nat} type satisfies the axioms of PA.

\paragraph*{The Calc Statement}

Before getting started, you should read section 21.17 of the Dafny manual to understand the basic usage of Calc statements, started with the keyword \texttt{calc}.

\paragraph*{Instructions}

In \texttt{PA.dfy}:
Provide the correct pre- and post-conditions for lemmas \emph{Zero} (2-1), \emph{Times zero} (2-3), \emph{Successor} (2-4), \emph{Plus successor} (2-5) and \emph{Time successor} (2-6),
and the bodies of your lemmas must satisfy each postcondition.
The proof for \emph{Additive identity} (2-2) is provided as an example.
If you use an \texttt{assume} statement in any of your lemmas, you will receive \emph{partial} (i.e. not full) credits.

Moreover, in this file:
Write a \emph{careful manual} proof for each of the lemmas above.
Again, the manual proof for \emph{Additive identity} (2-2) is  provided as an example.
By saying \emph{careful}, we mean that your proof must include all details and you should explain the reasons for each step.
Remember that the only thing you know are:
(1) the definitions of the functions listed in \texttt{PA.dfy}, and 
(2) a couple of ``built-in'' proof strategies (supported by Dafny and we admit them) including the (structural) induction, proof-by-contradiction and all equality axioms.

\paragraph{Proofs}

\subproblem{\emph{Zero}}

\begin{proof}
    \begin{itemize}
    \item If x = 0, by definition of 0 and 1, $x + 1 = 0 + 1 = 1 \neq 0$\\
    \item Else if x is not 0, then x must be the successor of n, x = Succ(n).
          By definition of plus, $x + 1 = succ(n) + 1 = succ(n + 1) \neq 0$ 
	\end{itemize}
	  Therefore, $\lnot (x + 1 = 0)$ holds for every $x$.

\end{proof}

\subproblem{\emph{Additive identity} (example)}

\begin{proof}
  By contradiction. Suppose that $x + 0 = x$ does not held for some $x$. Then, there are only two possible choices:
  \begin{itemize}
    \item $x$ is zero, i.e. $x = 0$. \\By definition of $+$, we know that $x + 0 = 0 + 0 = 0$. 
    Since $0 = x$, we conclude that $x + 0 = x$, contradiction!
    \item $x$ is the successor of $n$, i.e. $x = n + 1$. \\By definition of $+$, we know that $x + 0 = (n + 1) + 0 = (n + 0) + 1$. 
    By inductive hypothesis, $n + 0 = n$. Thus, $(n + 0) + 1 = n + 1 = x$. We again conclude that $x * 0 = 0$, contradiction!
  \end{itemize}
  Therefore, $x + 0 = x$ holds for every $x$.
\end{proof}

\subproblem{\emph{Times zero}}

\begin{proof}
    \begin{itemize}
      \item $x$ is zero, i.e. $x = 0$.\\ By definition of $\times$, we know that $x \times 0 = 0 \times 0 = 0$. 
      Since $0 = x$, we conclude that $x \times 0 = 0$!
      \item $x$ is the successor of $n$, i.e. $x = n + 1$. \\By definition of $\times$, we know that $x \times 0 = (n \times 0) + 0$. 
      By inductive hypothesis, $n \times 0 = 0$. Thus, $(n \times 0) + 0 = 0 + 0 = 0$. We again conclude that $x + 0 = x$!
    \end{itemize}
    Therefore, $x \times 0 = 0$ holds for every $x$.
\end{proof}

\subproblem{\emph{Successor}}

\begin{proof}
    \begin{itemize}
        \item If x = 0, then y has 2 conditions:
            \begin{itemize}
                \item $y = 0\\, x + 1 = y + 1 = 1\\, x = y = 0$, right.
                \item else, $y = Succ(n)$, then the prerequesite is not right, since\\
                $y + 1\\ 
                = Succ(n) + 1\\
                = Succ(n + 1)$, by definition of plus\\
                but for x,\\
                $x + 1\\
                = 0 + 1\\
                = Succ(0)$, by lamma 2, which will be proved in task 6\\
                
                since$\lnot (n + 1 = 0)$,$y + 1 \leq x + 1$, the prerequesite is wrong, then the formula is right.
            \end{itemize}{}
                \item else, x = Succ(m),then y has 2 conditions:
                    \begin{itemize}
                    \item $y = 0$, the prerequesite is similarly wrong, then the formula is right.
                    \item else, $y = Succ(n)$, \\
                    $x + 1  \\$
                    $=Succ(m) + 1\\$
                    $=Succ(m + 1)$\\
                    $y + 1 \\
                    = Succ(n) + 1\\
                    = Succ(n + 1)$\\
                    Since $x + 1 = y + 1$, then $Succ(m + 1) = Succ(n + 1), m + 1 = n + 1$\\
                    By induction hypothesis, $m = n$. Since$x = Succ(m), y = Succ(n), x = y$
            \end{itemize}{}
    \end{itemize}
    Therefore, $(x + 1 = y + 1) \to x = y$ holds for every $x, y$.

\end{proof}

\subproblem{\emph{Plus successor}}

\begin{proof}
    \begin{itemize}
        \item $x$ is zero, i.e. $x = 0$. \\
        $x + (y + 1)\\
        = 0 + (y + 1)\\
        = y + 1$,By definition of $+$\\
        $= (0 + y) + 1$,By definition of $+$\\
        $= (x + y) + 1$
        \item $x$ is the successor of $n$,\\
        $x + (y + 1)\\
        = Succ(n) + (y + 1)\\
        = Succ(n + (y + 1))$,By definition of $+$\\
        $= Succ((n + y) + 1)$,By inductive hypothesis\\
        $= Succ(n + y) + 1$,By definition of $+$\\
        $= Succ(n) + y + 1$,By definition of $+$\\
        $= (x + y) + 1$
      \end{itemize}
      Therefore, $x + (y + 1) = (x + y) + 1$ holds for every $x, y$.
\end{proof}

\subproblem{\emph{Times successor}}

\begin{proof}
   \paragraph{lamma 2}First, we must prove that $x + 1 = Succ(x)$\\
   \begin{itemize}
        \item $x$ is zero, i.e. $x = 0$.\\
        $
        x + 1\\
        = 0 + 1\\
        = 1\\
        = Succ(0)\\
        = Succ(x)
        $
        \item $x$ is the successor of $n$,\\
        $
        x + 1\\
        = Succ(n) + 1\\
        = Succ(n + 1)\\
        = Succ(Succ(n))$, by inductive hypothesis\\
        $= Succ(x)$\\
    \end{itemize}
    Therefore,$x + 1 = Succ(x)$ is true for all x
    \paragraph{lamma of exchange}Second, we must prove that $x + (y + z) = (x + z) + y$\\
       \begin{itemize}
        \item $z$ is zero, i.e. $z = 0$.\\
        $
        (x + y) + z \\
        = (x + y) + 0
        = (x + y)$,by Additive identity\\
        $= (x + 0) + y$,by Additive identity\\
        $= (x + z) + y$
        \item $z$ is the successor of $n$,\\
        $
        (x + y) + z \\
        = (x + y) + Succ(n)\\
        = (x + y) + (n + 1)$,by lamma 2\\
        $= (x + y) + n + 1$, by Plus Successor\\
        $= (x + n) + y + 1$, by inductive hypothesis\\
        $= Succ((x + n) + y)$, by lamma 2\\
        $= Succ(x + n) + y\\
        = (x + n + 1) + y$, by lamma 2\\
        $= x + (n + 1) + y$, by Plus Successor\\
        $= x + Succ(n) + y$, by lamma 2\\
        $= (x + z) + y$
    \end{itemize}
    Therefore,$x + (y + z) = (x + z) + y$ is true for all x,y,z
    \paragraph{Proof}Last, we can prove that $~x \times (y + 1) = x \times y + x$\\
       \begin{itemize}
        \item $x$ is zero, i.e. $x = 0$.\\
        $
        x \times (y + 1)\\
        = 0 \times (y + 1)\\
        = 0$, by definition of $\times$\\
        $= 0 + 0$, by definition of $+$\\
        $= (0 \times y) + 0$, by definition of $\times$\\
        $= (x \times y) + x$
        \item $x$ is the successor of $n$,\\
        $
        x \times (y + 1)\\
        = Succ(n) \times (y + 1)\\
        = (n \times (y + 1)) + (y + 1)$, by definition of multiply\\
        $= (n \times y + n) + (y + 1)$, by inductive hypothesis\\
        $= (n \times y + n + y) + 1$, by plus successor\\
        $= ((n \times y) + y) + n + 1$, by exchange lamma\\
        $= (Succ(n) \times y) + n + 1$, by definition of multiply\\
        $= (Succ(n) \times y) + Succ(n)$, by lamma 2\\
        $= (x \times y) + x$
    \end{itemize}
    Therefore,$~x \times (y + 1) = x \times y + x$ is true for all x,y
\end{proof}

%% problem ends

\end{document}
