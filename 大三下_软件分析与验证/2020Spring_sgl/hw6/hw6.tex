%% compile with pdflatex or xelatex
\documentclass[11pt,a4paper]{article}

\usepackage{homework}
\usepackage[inference]{semantic}
\usepackage{xcolor}

\title{Homework 6}
\duedate{Jun 3, 2020}

% TODO your name and ID
\studentname{SHEN GUANLIN}
\studentid{2017013569}

\newcommand{\pair}[1]{\langle #1 \rangle}
\newcommand{\evalA}[2]{{\mathcal{A}\llbracket #1 \rrbracket}_{#2}}
\newcommand{\evalB}[2]{{\mathcal{B}\llbracket #1 \rrbracket}_{#2}}
\newcommand{\evalC}[3]{\pair{#1, #2} \Downarrow #3}
\newcommand{\evalCsig}[4]{\pair{#1, #2} \Downarrow \pair{#3, #4}}
\newcommand{\evalAS}[3]{\pair{#1, #2} \rightarrow_{\mathcal{A}} #3}
\newcommand{\evalBS}[3]{\pair{#1, #2} \rightarrow_{\mathcal{B}} #3}
\newcommand{\evalCS}[4]{\pair{#1, #2} \rightarrow \pair{#3, #4}}
\newcommand{\evalCSM}[4]{\pair{#1, #2} \rightarrow_{*} \pair{#3, #4}}

\let\implies\twoheadrightarrow
\newcommand{\norm}[1]{\textsf{norm}(#1)}
\newcommand{\err}{\textsf{err}}

\newcommand{\kword}[1]{{\color{blue} \textsf{#1}}}
\newcommand{\True}{\kword{true}}
\newcommand{\False}{\kword{false}}
\newcommand{\Skip}{\kword{skip}}
\newcommand{\If}{\kword{if}}
\newcommand{\Then}{\kword{then}}
\newcommand{\Else}{\kword{else}}
\newcommand{\Fi}{\kword{fi}}
\newcommand{\While}{\kword{while}}
\newcommand{\Do}{\kword{do}}
\newcommand{\End}{\kword{end}}
\newcommand{\Havoc}{\kword{havoc}}
\newcommand{\Assume}{\kword{assume}}
\newcommand{\Assert}{\kword{assert}}
\newcommand{\Repeat}{\kword{repeat}}
\newcommand{\Until}{\kword{until}}

\newcommand{\wlp}[2]{\textsf{wlp}(#1, #2)}

\begin{document}

\maketitle

\textit{Read the instructions below carefully before you start working on the assignment:}
\begin{itemize}
    \item Please typeset your answers in the attached \LaTeX~source file, compile it to a PDF,
    and finally hand the PDF to Tsinghua Web Learning \emph{before the due date}.
    \item Make sure you fill in your \emph{name} and \emph{Tsinghua ID},
    and replace all ``\texttt{TODO}''s with your solutions.
    \item Any kind of dishonesty is \emph{strictly prohibited} in the full semester.
    If you refer to any material that is not provided by us, you \emph{must cite} it.
\end{itemize}

\problem{Multiple Choice}

For each of questions below, four choices marked (A), (B), (C) and (D) are provided.
ONLY ONE of them is correct. Read the questions carefully and choose the correct answers.

\subproblem Which of the following is \emph{not} a loop invariant for the following IMP loop? $$\While~Y>0~\Do~Y:=Y-1;X:=X+1~\End$$

\begin{enumerate}[label=(\Alph*)]
    \item $X>10$.
    \item $Y>10$.
    \item $X+Y=Z$.
    \item $Z+Y<X$.
\end{enumerate}

\begin{solution}
    B
\end{solution}

\subproblem Which of the following is \emph{false} about the IMP program shown below?
\begin{align*}
    & X := 1; \\
    & \While~X > 0~\Do \\
	&\quad \If~N \le 100~\Then \\
	&\qquad N := N + 11; \\
	&\qquad X := X + 1 \\
	&\quad \Else \\
	&\qquad N := N - 10; \\
	&\qquad X := X - 1 \\
	&\quad \Fi \\
    & \End
\end{align*}

\begin{enumerate}[label=(\Alph*)]
    \item $X=0$ is a post condition.
    \item $X \ge 0$ is a loop invariant (for the while-loop).
    \item $N \le 111$ is a loop invariant (for the while-loop).
    \item The program may not terminate.
\end{enumerate}

\begin{solution}
    D
\end{solution}

\subproblem Let $[X=0]~\While~b~\Do~c~\End~[X=1]$ be a Hoare triple.
Which of the following is \emph{true}?

\begin{enumerate}[label=(\Alph*)]
    \item If $c$ is $X:=1$, then the Hoare triple is valid for some $b$.
    \item If $b$ is $\True$, then the Hoare triple is valid for some $c$.
    \item If $b$ is $X\not=1$, then the Hoare triple is valid no matter what $c$ is.
    \item The Hoare triple is always invalid no matter what $b$ and $c$ are.
\end{enumerate}

\begin{solution}
    A
\end{solution}

\subproblem Recall that two IMP programs (with havoc) $c_1$ and $c_2$ are \emph{behaviorally equivalent},
if for every states $\sigma$ and $\sigma'$, their big-step operational semantic evaluation relations satisfy $\evalC{\sigma}{c_1}{\sigma'} \iff \evalC{\sigma}{c_2}{\sigma'}$.
In which of the following are $c_1$ and $c_2$ behaviorally equivalent?

\begin{enumerate}[label=(\Alph*)]
    \item $c_1: X:=Y; Y:=X \qquad c_2: Y:=X; X:=Y$
    \item $c_1: \Skip \qquad c_2: \If~X>10~\Then~X:=0~\Else~\Skip~\Fi$
    \item $c_1: \Havoc~X; X:=10 \qquad c_2: X:=10$
    \item $c_1: \Havoc~X; \Havoc~Y \qquad c_2: \Havoc~Y$
\end{enumerate}

\begin{solution}
   C
\end{solution}

\subproblem Let $F$ be a CNF with four variables $x_1, x_2, x_3, x_4$.
We apply the DPLL algorithm (without backjump) on $F$ and the following operations are done:
decide $x_1$, propagate $x_2$, propagate $x_3$.
Which of the following operations will be possibly done in the next step?

\begin{enumerate}[label=(\Alph*)]
    \item Decide $\lnot x_3$.
    \item Backtrack and decide $\lnot x_1$.
    \item Backtrack and decide $\lnot x_2$.
    \item Backtrack and decide $\lnot x_3$.
\end{enumerate}

\begin{solution}
   C
\end{solution}

\newpage
\problem{Assumptions \& Assertions}

We consider two kinds of commands which indicate a certain statement should hold any time this part of the program is reached -- the assumption statement ``$\Assume~b$'', and the assertion statement ``$\Assert~b$'':
\begin{itemize}
	\item If an assertion statement fails, it causes the program to go into an \emph{error state} and exit (or abort).
	\item If an assumption statement fails, the program fails to evaluate at all.
	In other words, the program gets \emph{stuck} and has no final state.
\end{itemize}
To formally express the program may go into an error state, we have to change the evaluation relation (of big-step operational semantics) from ``$\evalC{\sigma}{c}{\sigma'}$'' into ``$\evalC{\sigma}{c}{r}$'', where the evaluation \emph{result}
$$r ::= \norm{\sigma} \mid \err$$
can state two possible cases: $\norm{\sigma}$ for normally execution with ending state $\sigma$,
or $\err$ for reaching the error state.
The inference rules for the original IMP commands need be modified and we should handle errors carefully (read and think about the differences):
\begin{align*}
	&\inference[(Skip)]{}{\evalC{\sigma}{\Skip}{\norm{\sigma}}}
	&\inference[(Ass)]{\evalA{a}{\sigma}=n}{\evalC{\sigma}{x:=a}{\norm{\sigma[x\mapsto n]}}} \\
	&\inference[(Seq)]{\evalC{\sigma}{c_1}{\norm{\sigma'}} & \evalC{\sigma'}{c_2}{r}}
		{\evalC{\sigma}{c_1;c_2}{r}}
	&\inference[(SeqErr)]{\evalC{\sigma}{c_1}{\err}}{\evalC{\sigma}{c_1;c_2}{\err}} \\
	&\inference[(IfTrue)]{\evalB{b}{\sigma}=\top & \evalC{\sigma}{c_1}{r}}
		{\evalC{\sigma}{\If~b~\Then~c_1~\Else~c_2~\Fi}{r}}
	&\inference[(IfFalse)]{\evalB{b}{\sigma}=\bot & \evalC{\sigma}{c_2}{r}}
		{\evalC{\sigma}{\If~b~\Then~c_1~\Else~c_2~\Fi}{r}} \\
	&\inference[(WhileFalse)]{\evalB{b}{\sigma}=\bot}
		{\evalC{\sigma}{\While~b~\Do~c~\End}{\norm{\sigma}}}
	&\inference[(WhileTrue)]{\evalB{b}{\sigma}=\top & \evalC{\sigma}{c}{\norm{\sigma'}} \\
		\evalC{\sigma'}{\While~b~\Do~c~\End}{r}}{\evalC{\sigma}{\While~b~\Do~c~\End}{r}} \\
	&\inference[(WhileTrueErr)]{\evalB{b}{\sigma}=\top & \evalC{\sigma}{c}{\err}}
		{\evalC{\sigma}{\While~b~\Do~c~\End}{\err}} \\
\end{align*}

We redefine Hoare triples ``$\{P\}~c~\{Q\}$'' to mean that, whenever $c$ is started in a state satisfying $P$, and terminates with result $r$, then $r=\norm{\sigma}$ (and hence $r\not=\err$) where the state $\sigma$ satisfies $Q$.

\subproblem Give the evaluation rules for assumption and assertion statements.

\begin{solution}
    \begin{align*}
        &\inference[(Assume)]{\evalB{b}{\sigma}=\top}{\evalC{\sigma}{\Assume~b}{\norm{\sigma}}} \\	
        &\inference[(AssertTrue)]{\evalB{b}{\sigma}=\top}{\evalC{\sigma}{\Assert~b}{\norm{\sigma}}} \\
        &\inference[(AssertFalse)]{\evalB{b}{\sigma}=\bot}{\evalC{\sigma}{\Assert~b}{err}} \\	
    \end{align*}
\end{solution}

\subproblem Design Hoare rules for assumption and assertion statements.

\begin{solution}
   \begin{align*}
	&\inference[(Assume)]{\exists A. A \vDash P \land b}
	{\{P\}\Assume~b\{P\land b\}}
	&\inference[(Assert)]{P \Rightarrow b}
		{\{P\}\Assert~b\{P\}}
   \end{align*}
\end{solution}

\subproblem Compute $\wlp{X := X + 1; \Assume~X > 0; Y := Y + X}{X + Y + Y \ge 3}$.

\begin{solution}
    $\wlp{X := X + 1; \Assume~X > 0; Y := Y + X}{X + Y + Y \ge 3}$\\
     $=\wlp{X := X + 1; \Assume~X > 0}{\wlp{Y := Y + X}{X + Y + Y \ge 3}}$\\
     \\
    $\wlp{Y := Y + X}{X + Y + Y \ge 3} = {X + Y + X + Y + X\ge 3} $\\
    \\
    $\wlp{X := X + 1; \Assume~X > 0}{\wlp{Y := Y + X}{X + Y + Y \ge 3}}$\\
    $=\wlp{X := X + 1; \Assume~X > 0}{X + Y + X + Y + X\ge 3}$\\
     $=\wlp{X := X + 1} {\wlp{\Assume~X > 0}{X + Y + X + Y + X\ge 3}}$\\
     \\
     $\wlp{\Assume~X > 0}{X + Y + X + Y + X\ge 3} = X > 0 \to X + Y + X + Y + X\ge 3$\\
     \\
     $\wlp{X := X + 1} {\wlp{\Assume~X > 0}{X + Y + X + Y + X\ge 3}}$\\
     $=\wlp{X := X + 1}{ X > 0 \to X + Y + X + Y + X\ge 3}$\\
     $=(X+1) > 0 \to (X+1) + Y + (X+1) + Y + (X+1) \ge 3$\\\
     To sum up, the wlp is  $=(X+1) > 0 \to (X+1) + Y + (X+1) + Y + (X+1) \ge 3$
     
\end{solution}

%% problem ends

\end{document}
