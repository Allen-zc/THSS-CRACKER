\documentclass{article}
\usepackage{CJKutf8}
\usepackage{latexsym}
\title{Week 1}
\author{软73 沈冠霖 2017013569}
\begin{document}
\begin{CJK}{UTF8}{gkai}
\maketitle

\section{题目思路与证明}
\paragraph{1.1 BitAnd} 
因为对于整数x,y的任意一个二进制位,设为a,b,根据摩根定律,有$(a \& b) = \sim ((\sim a) | (\sim b))$,且其每一位运算相互独立,因此 $x\&y = \sim((\sim x)|(\sim y))$ 成立\\
\paragraph{1.2 GetByte} 
只需要做两件事:\\
首先,把x右移8n位,让你想要的8个bit落在x的最后8位.\\
其次,用和0xff按位且运算提取这8位\\
\paragraph{1.3 LogicalShift} 
问题核心:构造前n位为1,其余为0的数\\
对于非负数,直接右移就是正确结果,而对于负数,右移n位后要消去前n位的1\\
因此,我先获取x的第一位,signal,再将其右移n-1位\\
如果x为非负数,则signal = 0,右移n-1位后还是0;\\
如果x为负数,则signal第一位是1,其余为0,右移n-1位后,其前n位是1,其余是0\\
将x与signal做异或,就可以保留后面的信息,同时消除前n位可能的1\\

\paragraph{1.4 BitCount} 
借鉴了 https://www.douban.com/note/274239939/ 的一种思路 \\
先考虑2位的情况:a[0]+a[1] = a[0] + ($a>>1$)[0] =  2位里1的个数\\
计算出每个2位数字中1的个数之后,可以把每4位中的左两位右移2位,和右两位相加,得到每4位数字中1的个数\\
以此类推,计算出每$2^{n}$位数字中1的个数之后,可以把前$2^{n}$位右移n位,和后$2^{n}$位相加,就可以得到$2^{(n+1)}$位中的个数\\

\paragraph{1.5 Bang} 
问题的核心:0的相反数还是0,其符号位都是0;\\
其余$\forall x != TMIN$,有$sign(x)!=sign(-x)$\\
TMIN和-TMIN都是0x80000000,符号都是1\\
因此只需要求出x和$-x=\sim x+1$两个数的符号,做或运算就可以了

\paragraph{2.1 tmin} 
如果x是长度为n字节的two's complement integer,则 $-2^{n-1}\leq x \leq 2^{n-1}-1$,
对于32位的int,最小的这个数显然是$INT\_MIN$,也就是$1<<31$\\
\paragraph{2.2 fitsBits} 
如果n=32,那么任何int范围内的数都是\\
如果其他情况,那么只要$0\leq x + 2^{n-1} \leq 2^{n}-1$,x也是.\\
所以我先判断n是否为32,如果不是的话,就判断x加上$2^{n-1}$这个偏置之后是否在此范围内\\

\paragraph{2.3 divpwr2} 
为了实现向0取整,应该这样:负数向上偏移$2^{n} -1$,非负数不变,也就是向上偏移$2^{0} -1$\\
因此,我提取了符号位并且让字符signal的每一位都是符号位,然后偏移$2^{n\&signal}-1$\\
这样如果是非负数,则n\&signal = 0,否则n\&signal = n,就能实现了\\

\paragraph{2.4 negate} 
因为每个int都是补码存储,所以其逐位取反再+1就是其相反数

\paragraph{2.5 IsPositive} 
若x是负数,则首位是1,!x = 0;\\
若x = 0,!x = 1,首位是0;\\
若x是正数,则首位是0,!x = 0\\
因此我计算 !x + 首位的结果,然后取非就可以判断了\\

\paragraph{2.6 IsLessOrEqual} 
先单独考虑x=y,此时x-y=0,!(x-y)=1
而再考虑$x<y$,此时,如果x与y同为非负数或者同为负数,x - y必定不溢出,结果正负反映了其大小关系\\
如果x,y符号位不同,则可以之间判断其符号位.\\
因此我设x的符号位为二进制变量a,y的符号位为b,$x-y$的符号位为c,则对于x!=y的情况,可以画出如Table.1的卡诺图:\\
根据卡诺图,可以化简得到:结果$ans = (a\&(\sim b))|(a\&c)|((\sim b)\&c)$\\

\begin{table}[!htbp] 
	
	\caption{2.6题的卡诺图求解}
	\begin{flushleft} 
		\begin{tabular}{|l|l|l|l|l|} 
			\hline   & 00 & 01 & 11 & 10\\ 
			\hline 0 &0 &0 &0 &1 \\ 
			\hline 1 &1 &0 &1 &1 \\
			\hline
		\end{tabular} 
		注:第一行代表了ab的二进制取值,第一列代表了c的二进制取值,表中其他数据代表了abc不同取值情况下的二进制输出
	\end{flushleft} 
\end{table}

\paragraph{2.7 ILog2} 
这道题的实质是求从左到右数第一个1的位置,和1.5一样,这道题还是要看每一位,因此也模仿1.5使用递归.\\
思路如下:如果前16位有1(不是0),只需要在前16位找1就可以了,否则只在后16位找\\
当前的16位,如果前8位有1,只需要在前8位找,否则只在后8位找\\
以此类推,宏观上来说,可以使用5层递归来完成任务.\\
微观上来说,需要一个二进制变量w记录前一半位置是否为0,并且找到一个函数f(w),使得当w=1时,使用后一半的数字,并且,w=0时,使用前一半的数字\\
因此,我用current来记录当前需要处理的一串数字,用place记录那串数字最右一位的位置(比如说,我处理前16位,current就是$x>>16$,place就是16).\\
令f(w)=$\sim w+1$,这样当w=1时,f(w)全是1;w=0时,f(w)全是0\\
$new\_current$ = current前半部分 + current后半部分\&f(w),这样,如果w=1,前半部分是0,后半部分留了下来;如果w=0,前半部分留了下来,后半部分被消除\\
place += $\sim $f(w)\&当前一半的位数,这样,如果w=0,则place向左进一半的位数,否则不动\\


\paragraph{3.1 FloatNeg} 
只需要判断是否是NaN,如果是NaN,也就是exp位都是1,后23位不都是0,则直接输出\\
否则,翻转符号后输出\\
\paragraph{3.2 FloatI2f} 
首先,要提取x的符号,把x转化为其绝对值.其中,0和$MIN\_INT$为特例,需要提前识别输出.\\
之后,要用移位法计算出x绝对值最高为1的位的位置(对应2的几次方),再加上偏置变成指数\\
最后,需要提取出x的小数部分,也就是x绝对值排除掉最高位的其他位,移到对应位置.\\
如果x的小数部分有需要舍弃的位的话,需要将其提取出来.\\
设x小数部分保留的最低位是a,被x舍弃的小数部分是b,根据保留小数的规则,分三种情况:\\
1:$b < 0.5$,小数部分不变\\
2:$b > 0.5$,小数部分+1\\
3:b = 0.5,此时如果a = 1,则小数部分+1,来向偶数取整;否则小数部分不变.\\

\paragraph{3.3 FloatTwice} 
首先,先用位运算提取符号位,指数位,小数位\\
之后,根据当前的指数位分为三种情况:1:这个数是无穷或NaN,直接输出\\
2:这个数是normalized,将指数位+1,如果此时指数位变成全1,则输出无穷\\
3:这个数是denormalized,指数位都是0的数,小数位需要左移一位.如果原来的小数位第一位是1,则这个数变为正规数,指数位变为1\\

\section{总结} 
这些题目大概能归纳出如下的思路:\\
\paragraph{1.提取特征}类似1.5,2.6题,都有明显的特征:输出是0或1,都是把情况进行分类。这种题我的思路通常是提取值的特征,比如正负数的符号,0与0x80000000与其相反数完全相同之类的,然后用这些特征的逻辑关系求解。对于较少的特征,可以直接列出表达式,如果像2.6这种三四个特征,可以使用数电课上学习过的卡诺图法来化简逻辑表达式,以最小化算符。
\paragraph{2.提取位置}像1.2这种题,还有很多题的中间步骤需要提取一个数的某些位置,提取的方法通常是:先把这个数右移到0,再用一个辅助数(要的位是1,否则是0)和这个数做与运算,以排除干扰。\\
\paragraph{3.排除负数干扰}我总结出,这些题中,负数有两种干扰方式。首先,负数的右移会让前几位都是1(1.3,1.4,2.3).其次,浮点数的运算算的是绝对值,和负数关系不大。\\
我的解决方案有三种:首先,如果要进行不断的移位处理,我会先求出符号位,消除符号位,然后再操作。其次,对于除法取整问题,可以向上偏移一个单位。最后,对于float操作,需要将其符号位求出来,转化成绝对值。
\paragraph{4.递归/分治/二分法}有些问题,比如1.4,2.6,需要操作一个二进制数的每一位才行,这种情况下,操作时间复杂度很高,需要优先考虑分治/二分法,把$\theta(n)$的问题转化为$\theta(lgn)$\\
\paragraph{5.浮点数的问题}浮点数严格分为三部分,因此,对于浮点数,我的一般处理方法是严格分为三个部分进行处理。而浮点数还有两种特殊情况:指数位全1的情况(无穷与溢出),指数位全0的非正规情况以及其和正规情况的相互转化。


\end{CJK}
\end{document}
